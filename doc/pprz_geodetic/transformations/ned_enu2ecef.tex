With a know transformation matrix (see section \ref{LTP}) it is quite easy to rotate a vector from the ENU-frame to the ECEF-frame.
Since
\begin{equation}
\mat R_{enu2ecef} = \inv{\mat R_{ecef2enu}} = \transp{\mat R_{ecef2enu}}
\end{equation}
a transformation is done as follows
\begin{equation}
\vect v_{ECEF} = \mat R_{enu2ecef} \vect v_{ENU} 
\end{equation}
For a transformation from the NED-frame you have to do an additional ENU/NED-transformation before.

\inCfile{ecef\_of\_enu\_vect\_d(EcefCoor\_d* ecef, LtpDef\_d* def, EnuCoor\_d* enu)}{pprz\_geodetic\_double}
\inCfile{ecef\_of\_ned\_vect\_d(EcefCoor\_d* ecef, LtpDef\_d* def, NedCoor\_d* ned)}{pprz\_geodetic\_double}

The transformation of a point is very similiar. After transforming into the ECEF-frame you add the position of the center  $p_0$ to the result.
\begin{equation}
\vect p_{ECEF} = p + p_0
\end{equation}
\inCfile{ecef\_of\_enu\_point\_d(EcefCoor\_d* ecef, LtpDef\_d* def, EnuCoor\_d* enu)}{pprz\_geodetic\_double}
\inCfile{ecef\_of\_ned\_point\_d(EcefCoor\_d* ecef, LtpDef\_d* def, NedCoor\_d* ned)}{pprz\_geodetic\_double}